%============ 대회 규정 페이지 입니다. ============
\newpage
\begin{spacing}{1.15}

\begin{center}
    \section*{대회 일정 및 규정 안내}
    
    일정 및 규정 미숙지로 인해 발생한 불이익의 책임은 참가자에게 있습니다.
\end{center}

\subsection*{대회 일정}

\begin{table}[h!]
    % \centering
    \renewcommand{\arraystretch}{1.7}
    \begin{tabular}{|C{3cm}|L{12cm}|}
    \hline
    \multicolumn{1}{|c}{일정} & \multicolumn{1}{|c|}{내용} \\
    \hline
    \hline
    12:00-12:30 & 개회식 및 \textbf{오상윤} 지도 교수님 소개\\ \hline
    12:30-13:20 & 후원사 소개 - 스타트링크, 솔브드, 삼성 디스플레이, 현대모비스, MORAI, DSPACE\\ \hline
    13:20-14:00 & 대회장 이동 및 환경점검\\ \hline
    14:00-18:00 & 본 대회 \\ \hline
    18:00-19:00 & 시상식 및 문제 해설 \\ \hline
    19:00- & (비공식)운영진과 함께하는 \textbf{\color{pink}오붓한} 저녁식사 \\ \hline
    \end{tabular}
\end{table}

\subsection*{\color{red}2023년 개정된 규정}
\begin{itemize}
    \item 하드카피, 소프트카피 관련 규정이 개정되었습니다.
    
    \begin{itemize}
        \item 하드카피, 소프트카피는 허용되지 않음 $\Rightarrow$ 소프트카피\textsuperscript{인쇄되지 않은 디지털 문서 파일}는 허용되지 않음, 하드카피\textsuperscript{물리적으로 인쇄된 문서}는 앞, 뒷면 상관 없이 최대 25장 허용(A4용지 기준) %A4 수정
    \end{itemize}
    
    \item Division2 참가 자격이 변경되었습니다.
    \begin{itemize}
        \item 정보통신대학 및 소프트웨어융합대학 재학생을 제외한 모든 재학생 $\Rightarrow$ 정보통신대학 및 소프트웨어융합대학 재학생을 제외한 모든 재학생 및 정보통신대학 및 소프트웨어융합대학 1학년 재학생
        \item 예시
        \begin{itemize}
            \item 소프트웨어학과 3,4학년\tabto{4cm} Div.1만 참가 가능
            \item 사이버보안학과 1학년 \tabto{4cm} Div.1,2 모두 참가 가능
            \item e-비즈니스학과 $N$학년 \tabto{4cm} Div.1,2 모두 참가 가능
        \end{itemize}
    \end{itemize}
\end{itemize}

\subsection*{대회 참가 자격}

\begin{itemize}
    \item 2023 APC 당일 아주대학교에 소속된 자
    \begin{itemize}
        \item 휴학생은 상금 수령은 불가하나, 시상 및 shake! 출전권 획득은 가능합니다.
    \end{itemize}
    \item 역대 한국 대학생 프로그래밍 경시대회 수상자가 아닌 자
    \item 역대 shake! 3위 이상 수상자가 아닌 자
    \item 역대 APC Division1 수상자는 시상에서 제외되며, shake! 출전권은 획득 가능
    \item 역대 APC Division2 수상자는 Division1에만 참가 가능
\end{itemize}


\subsection*{대회 주요 규칙}

\textbf{대회 진행 전}

\begin{itemize}
    \item APC는 1인 1팀의 개인전 형태로 치러집니다.
    \item 대회에 사용할 컴퓨터는 개인 지참해야 하며, 1인당 1대의 컴퓨터만 허용합니다.
    \item 개인 컴퓨터에는 대회 시작 전 컴파일 할 수 있는 환경을 스스로 준비해야 합니다. IDE 사용에 제한은 없습니다. (단, 온라인 IDE를 사용할 경우 해당 코드가 공개된다면 부정행위로 간주합니다.)
    \item 개인 컴퓨터와 마우스, 키보드를 제외한 모든 전자기기의 사용은 금지됩니다. 마우스와 키보드는 편하신 것으로 지참 가능합니다.
    \item 소프트 카피\textsuperscript{인쇄되지 않은 디지털 문서 파일}는 허용되지 않습니다.
    \item 하드 카피\textsuperscript{물리적으로 인쇄된 문서}는 허용되며 앞, 뒷면 상관 없이 최대 25장 까지 가능합니다. 사전에 인쇄해서 대회 당일 가져오신 후 스태프에게 검사를 맡은 후 사용해 주시면 됩니다.
지참 가능한 인쇄물의 예시는 다음 링크에서 참고하실 수 있습니다: \href{https://www.acmicpc.net/board/view/21870}{https://www.acmicpc.net/board/view/21870}
\end{itemize}

\textbf{대회 진행 중}

\begin{itemize}
    \item 타인과 의견을 주고받을 수 있는 모든 메신저의 사용은 금지됩니다.
    \item 대회가 진행되는 동안 타인 간에 의사소통, 자료 공유 등의 모든 행위는 금지됩니다.
    \item 대회 시작 후 1시간 경과 이전까지는 대회장을 나가실 수 없습니다. 또한 대회장을 나가신 이후에는 더 이상의 풀이 제출은 하실 수 없습니다.
    \item 아래 안내되는 언어별 레퍼런스 페이지를 제외한 모든 웹 페이지 접속은 금지됩니다. 
    \begin{itemize}
        \item \texttt{C/C++} \tabto{2cm} \href{https://en.cppreference.com/w/}{https://en.cppreference.com/w/}
        \item \texttt{Java} \tabto{2cm} \href{https://docs.oracle.com/javase/8/docs/api/}{https://docs.oracle.com/javase/8/docs/api/}
        \item \texttt{Python} \tabto{2cm} \href{https://docs.python.org/3/}{https://docs.python.org/3/}
    \end{itemize}
\end{itemize}

\textbf{기타}

\begin{itemize}
    \item 본인의 아이디와 패스워드를 타인에게 공개하거나 온/오프라인 상에서 공개하는 행위는 부정행위로 간주합니다.
    \item 모든 부정행위자는 실격 처리 되며, 그 결과를 각 학과에 통보합니다.
    \item 참가 신청 후 별도의 통보 없이 대회장에 미참석 하는 경우 불이익이 있을 수 있습니다.
    \item 만일 피치 못할 사정으로 대회 당일 참석하지 못할 경우 사전에 주관처(\href{mailto: ansi.ajou@gmail.com}{ansi.ajou@gmail.com}) 혹은 기타 운영진을 통해 연락 부탁드립니다.
\end{itemize}

\subsection*{대회 진행 방식}

\begin{itemize}
    \item 문제의 모든 지문은 한국어로만 제공됩니다.
    \item 참가자는 각 문제에 대한 해답을 작성하는 소스코드를 제출합니다.
    \item 사용가능한 프로그래밍 언어는 C/C++, Java, Python3, PyPy3로 제한됩니다. 순위 책정에 언어의 종류는 관계가 없습니다. 
    \item 제출된 소스코드는 시스템에 의해 실시간으로 채점됩니다.
    \item 채점이 완료되면 참가자는 채점 결과를 확인할 수 있습니다.
    \item 제출 횟수에 제한은 없습니다.
    \item 대회 중 문제 및 채점에 관한 질문은 대회 진행 페이지의 문의하기 기능을 통해 문의해야 합니다.
    \item 문제와 채점에 관련되지 않은 사항은 대회장에 있는 감독관에게 직접 문의합니다.
    \item 대회 중 심각한 오류가 발견된 경우 문제의 수정 및 재채점이 가능하며, 이는 모든 참가자에게 공지됩니다.
\end{itemize}

\subsection*{순위 결정 방식}

\begin{itemize}
    \item 모든 참가자는 실시간 순위를 직접 확인할 수 있습니다.
    \item 모든 부정행위자는 순위에서 제외됩니다.
    \item Division1 - Competition Round와 Division2 - Challenge Round는 독립적으로 순위를 적용합니다.
    \item 모든 참가자는 소속, 학년, 나이, 성별 등 대회 외적인 사항으로 결과에 이익이나 불이익을 받지 않습니다.
    \item 각 참가자는 문제를 풀어 획득한 점수와 패널티를 가집니다.
    \item 제출한 답이 해당 문제 최초 정답인 경우, 제출자에게 (대회 경과 시간) + (해당 문제 오답 제출 수 x 20분)의 패널티를 가산합니다.
    \item 정답을 맞추지 못한 문제에 대한 패널티는 부가되지 않습니다.
    \item 순위는 아래 조건을 순차적으로 적용했을 때, 상위에 있는 조건을 먼저 만족한 참가자가 더 높은 순위를 가집니다.
\end{itemize}

\subsection*{채점 환경}

\begin{itemize}
    \item 모든 채점은 Startlink의 백준 온라인 저지 플랫폼에서 이루어지며, 참가자들은 미리 플랫폼을 이용해볼 수 있습니다.
    \item 언어마다 다른 채점기준이 적용되며, 각 언어별 컴파일 및 실행 옵션, 버전, 채점기준은 언어별 예제 등의 자세한 사항은 \href{https://help.acmicpc.net/language/info}{여기}에서 확인 바랍니다.
\end{itemize}

\begin{table}[h!]
    \centering
    \renewcommand{\arraystretch}{1.4}
    \begin{tabular}{|L{1.7cm}|L{14.5cm}|}
    \hline
    \multicolumn{1}{|c}{언어} & \multicolumn{1}{|c|}{버전}\\
    \hline
    \hline
    \texttt{C11} & \texttt{gcc (GCC) 11.1.0}\\ \hline
    \texttt{C++17} & \texttt{g++ (GCC) 11.1.0}\\ \hline
    \texttt{Java} & \texttt{Java(TM) SE Runtime Environment (build 1.8.0\_201-b09)}\\ \hline
    \texttt{Python3} & \texttt{Python 3.11.0}\\ \hline
    \texttt{PyPy3} & \texttt{Python 3.9.12, PyPy 7.3.9 with GCC 10.2.1 20210130 (Red Hat 10.2.1-11)}\\ \hline
    \end{tabular}
\end{table}

\end{spacing}