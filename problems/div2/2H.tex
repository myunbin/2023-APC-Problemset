\newpage
\section*{{\Large 문제 H.} \tabto{2cm}{\LARGE 뭐라고? 안들려}}

\begin{itemize}
    \item 시간 제한 \tabto{2cm} 2초
\end{itemize}

\hrule

\subsection*{문제}

2차원 좌표 평면상에 현빈이와 수연이가 살고 있다. 현빈이와 수연이는 통화를 자주 하는데, 둘 다 오래된 핸드폰을 쓰기 때문에 통화가 자주 끊긴다. 둘은 이리저리 자리를 옮기며 통화하던 중, 둘의 위치를 잇는 직선의 기울기가 $K$라면 통화가 끊기지 않는다는 사실을 발견했다.

현빈이와 수연이가 있을 수 있는 $N$개의 2차원 좌표가 주어질 때, 통화가 끊기지 않도록 현빈이와 수연이를 배치하는 경우의 수를 구해주자.

\subsection*{입력}

첫째 줄에는 $N$과 $K$가 공백으로 구분되어 주어진다. $(2\leq N\leq 200\,000;$ $-10^{9}\leq K \leq 10^{9})$

둘째 줄부터 $N$개 줄에 걸쳐 $i$번 점의 $x$좌표와 $y$좌표가 공백으로 구분되어 주어진다. $(-10^{9}\leq x_{i},y_{i} \leq 10^{9})$

입력으로 주어지는 모든 값은 정수다.

\subsection*{출력}

통화가 끊기지 않도록 현빈이와 수연이를 배치하는 경우의 수를 구하여 출력하시오.

\subsection*{예제}

\begin{table}[h]
% \centering
\renewcommand{\arraystretch}{1.5}
\begin{tabular}{|L{8.2cm}|L{8.2cm}|}
\hline
\multicolumn{1}{|c|}{\textbf{standard input}} & \multicolumn{1}{c|}{\textbf{standard output}} \\ \hline\hline
% 적절한 예제를 입력하면 됩니다.
\texttt{8 1} & \texttt{16}\\ 
\texttt{-1 -1} & \\
\texttt{2 3} & \\
\texttt{3 4} & \\
\texttt{0 0} & \\
\texttt{5 5} & \\
\texttt{3 1} & \\
\texttt{-7 -9} & \\
\texttt{6 6} & \\
\hline
\end{tabular}
\end{table}

만약 현빈이가 $(0,0)$, 수연이가 $(6,6)$에 있다면 두 점을 잇는 직선의 기울기가 $1$이기 때문에 둘의 통화는 끊기지 않는다.

\newpage

\vspace*{10cm}
\begin{center}
    {\Huge \color{gray} 여백의 미} 
\end{center}