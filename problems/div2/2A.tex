\newpage
\section*{{\Large 문제 A.} \tabto{2cm}{\LARGE Since 1973}}

\begin{itemize}
    \item 시간 제한 \tabto{2cm} 1초
\end{itemize}

\hrule

\subsection*{문제}

\begin{figure}[h]
    \centering
    \includegraphics[width=0.3\textwidth]{problems/image/50th_emblem_ver02.png}
    \caption{\textbf{아주대학교는 1973년에 개교하여 올해로 개교 50주년을 맞이했다.}}
\end{figure}

이에 기뻐한 선우는 어떤 수에 $50$이 들어가면 그 숫자를 한 번 더 세기로 하였다.

예를 들어 $5152$는 한번만 세지만 $5051$은 한 번 더 센다.

어떤 수 $N$이 주어질 때, 선우의 방식대로 세면 몇 번째의 숫자인지 구하여라. 선우의 방식대로라면 $50$은 $50$번째 숫자지만, $51$은 $52$번째 숫자다.

\subsection*{입력}

$N$이 주어진다. $(1 \leq N\leq 1\, 000\, 000)$

\subsection*{출력}

선우의 방식대로 셌을 때 어떤 수 $N$이 몇 번째 숫자인지 출력하시오.

\subsection*{예제}

\begin{table}[h]
% \centering
\renewcommand{\arraystretch}{1.5}
\begin{tabular}{|L{8.2cm}|L{8.2cm}|}
\hline
\multicolumn{1}{|c|}{\textbf{standard input}} & \multicolumn{1}{c|}{\textbf{standard output}} \\ \hline\hline
% 적절한 예제를 입력하면 됩니다.
\texttt{50} & \texttt{50}\\ 
\hline

\texttt{51} & \texttt{52}\\ 
\hline

\end{tabular}
\end{table}

\subsection*{노트}
아주대학교의 50살을 축하합니다! 그렇다면 APC는 몇살일까요?

\newpage

\vspace*{10cm}
\begin{center}
    {\Huge \color{gray} 여백의 미} 
\end{center}