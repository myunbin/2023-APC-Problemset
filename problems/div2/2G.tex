\newpage
\section*{{\Large 문제 G.} \tabto{2cm}{\LARGE 2022 APC가 어려웠다고요?}}

\begin{itemize}
    \item 시간 제한 \tabto{2cm} 1초
\end{itemize}

\hrule

\subsection*{문제}

2022 APC는 출제진들의 생각보다 난이도가 어려웠다고 한다. 난이도의 조절을 위해 2023 APC의 출제자 현빈이는 문제 난이도에 대한 가이드라인을 만들었다. 2023 APC에는 $N$개의 문제가 출제될 예정이다. 각 문제는 순서대로 $1$번부터 $N$번까지의 번호를 가진다.

현빈이가 만든 난이도 가이드라인에는 두 가지 규칙이 있다.

\begin{itemize}
    \item $i$번 문제의 난이도 $D_{i}$는 $a_{i}\leq D_{i}\leq b_{i}$를 만족하는 정수다.
    \item 인접한 문제의 난이도 차이는 $K$이하다. 다시 말해 모든 $i$에 대해 $|D_{i}-D_{i+1}|\leq K$를 만족한다.
\end{itemize}

현빈이를 위해 난이도 가이드라인의 규칙을 준수하면서 $N$개 문제의 난이도를 결정하는 방법의 수를 구해주자.

\subsection*{입력}

첫째 줄에 $N$과 $K$가 주어진다. $(1\leq N\leq 3\,000; 0\leq K \leq 3\,000)$

둘째 줄부터 $N$줄에 걸쳐 $a_{i}$와 $b_{i}$가 공백으로 구분되어 주어진다. $(1\leq a_{i} \leq b_{i} \leq 3\,000)$

입력으로 주어지는 모든 값은 정수다.

\subsection*{출력}

문제의 조건에 맞게 $N$개 문제의 난이도를 결정하는 방법의 수를 $10^{9}+7$로 나눈 나머지를 출력하시오.

\subsection*{예제}

\begin{table}[h]
% \centering
\renewcommand{\arraystretch}{1.5}
\begin{tabular}{|L{8.2cm}|L{8.2cm}|}
\hline
\multicolumn{1}{|c|}{\textbf{standard input}} & \multicolumn{1}{c|}{\textbf{standard output}} \\ \hline\hline
% 적절한 예제를 입력하면 됩니다.
\texttt{4 3} & \texttt{140}\\ 
\texttt{1 7} & \\ 
\texttt{2 3} & \\ 
\texttt{4 6} & \\ 
\texttt{3 7} & \\ 
\hline

\texttt{3 0} & \texttt{3}\\ 
\texttt{1 7} & \\ 
\texttt{2 6} & \\ 
\texttt{3 5} & \\ 

\hline
\end{tabular}
\end{table}

\subsection*{노트}

죄성함다.

\newpage

\vspace*{10cm}
\begin{center}
    {\Huge \color{gray} 여백의 미} 
\end{center}
