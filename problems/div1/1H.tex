\newpage
\section*{{\Large 문제 H.} \tabto{2cm}{\LARGE 기벡을 안배운다고?}}

\begin{itemize}
    \item 시간 제한 \tabto{2cm} 2초
\end{itemize}

\hrule

\subsection*{문제}

민우는 22학번이다. 2022학년도 수능 수학에 선택 과목 제도가 생기면서 선택 과목으로 미적분을 택한 민우는 기하와 벡터 과목의 아름다움을 알지 못했다. 기하와 벡터를 독학으로 통달한 민우는 2023 APC를 통해서라도 기하와 벡터의 아름다움을 설파하고자 한다. 2차원 평면상의 두 벡터 $v_{1}=\left( x_{1},y_{1} \right)$와 $v_{2}=\left( x_{2},y_{2} \right)$의 내적(dot product)은 다음과 같이 정의된다.
$$v_{1}\cdot v_{2}=x_{1}x_{2}+y_{1}y_{2}$$

만약 두 벡터의 내적 값이 $0$이라면, 2차원 평면상에서 두 벡터는 수직임을 의미한다.

$N$개의 2차원 벡터가 주어질 때, $1\leq i<j\leq N$이면서 두 벡터 $v_{i}$와 $v_{j}$가 수직인 순서쌍 $(i,j)$의 개수를 구하시오.

\subsection*{입력}

첫째 줄에 $N$이 주어진다. $(1\leq N \leq 200\,000)$

둘째 줄부터 $N$개 줄에 걸쳐 벡터 $v_{i}=\left(x_{i},y_{i}\right)$의 $x_{i}$와 $y_{i}$가 공백으로 구분되어 주어진다. $(-10^{9}\leq x_{i},y_{i} \leq 10^{9})$

입력으로 주어지는 모든 값은 정수다.

\subsection*{출력}

$1\leq i < j \leq N$을 만족하는 정수 $i$와 $j$에 대하여, 두 벡터 $v_{i}$와 $v_{j}$가 서로 수직인 순서쌍 $(i,j)$의 개수를 출력하시오.

\subsection*{예제}

\begin{table}[h]
% \centering
\renewcommand{\arraystretch}{1.5}
\begin{tabular}{|L{8.2cm}|L{8.2cm}|}
\hline
\multicolumn{1}{|c|}{\textbf{standard input}} & \multicolumn{1}{c|}{\textbf{standard output}} \\ \hline\hline
% 적절한 예제를 입력하면 됩니다.
\texttt{9} & \texttt{10}\\ 
\texttt{1 2} & \\ 
\texttt{3 5} & \\ 
\texttt{3 6} & \\ 
\texttt{2 -1} & \\ 
\texttt{-4 2} & \\ 
\texttt{4 -2} & \\ 
\texttt{-5 0} & \\ 
\texttt{-8 -16} & \\ 
\texttt{0 3} & \\ 

\hline
\end{tabular}
\end{table}

\newpage

\vspace*{10cm}
\begin{center}
    {\Huge \color{gray} 여백의 미} 
\end{center}