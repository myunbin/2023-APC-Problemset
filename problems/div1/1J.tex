\newpage
\section*{{\Large 문제 J.} \tabto{2cm}{\LARGE 너의 집에 가까워졌어 너의 이름을 크게 불러봐도 너는 너무 멀어}}

\begin{itemize}
    \item 시간 제한 \tabto{2cm} 1초
\end{itemize}

\hrule

\subsection*{문제}

\begin{quote}
너의 집에 가까워졌어 너의 이름을 크게 불러봐도 너는 너무 멀어 

아무 의미 없어진 나의 산책 너가 묻은 길을 돌아보고 다시 길을 걸어
\qauthor{멀어 (feat. Beenzino) - Primary}
\end{quote}

현빈이와 수연이의 집은 너무 멀리 떨어져 있다. 이 둘이 사는 도시에는 $1$번 집부터 $N$번 집까지 총 $N$개의 집이 있다. 또, 서로 다른 두 집을 잇는 거리 $1$의 양방향 오솔길이 $N$개 있다. 임의의 두 집을 잇는 오솔길은 최대 한 개고, 임의의 두 집 사이에는 하나 이상의 오솔길을 이용하는 경로가 반드시 존재한다.

이 도시의 시장 민우는 오솔길 \textbf{하나}를 제거할 계획을 하고 있다. 민우는 현빈이와 수연이 같은 연인들의 왕래가 편했으면 한다. 따라서 오솔길 하나를 제거한 후에도 임의의 두 집 $u,v$ $(u<v)$ 사이의 경로가 존재하면서 모든 $(u,v)$ 쌍에 대한 거리의 합을 최소로 만들고 싶다.

민우와 이 도시의 연인들을 도와주자.

\subsection*{입력}

첫째 줄에 $N$이 주어진다. $(3\leq N \leq 200\,000)$

둘째 줄부터 $N$줄에 걸쳐 오솔길로 이어지는 두 집 $u$와 $v$가 공백으로 구분되어 주어진다. $(1\leq u,v \leq N; u\neq v)$ 

입력으로 주어지는 모든 값은 정수다.

\subsection*{출력}

문제의 조건을 만족하도록 한 개의 오솔길을 제거하였을 때 $\sum_{u<v}{d(u,v)}$의 최솟값을 출력하라. 

이때, $d(u,v)$는 $u$번 집과 $v$번 집 사이의 거리를 의미한다.

\newpage 

\subsection*{예제}

\begin{table}[h]
% \centering
\renewcommand{\arraystretch}{1.5}
\begin{tabular}{|L{8.2cm}|L{8.2cm}|}
\hline
\multicolumn{1}{|c|}{\textbf{standard input}} & \multicolumn{1}{c|}{\textbf{standard output}} \\ \hline\hline
% 적절한 예제를 입력하면 됩니다.
\texttt{8} & \texttt{62}\\ 
\texttt{1 2} & \\
\texttt{1 3} & \\
\texttt{2 3} & \\
\texttt{3 4} & \\
\texttt{2 5} & \\
\texttt{2 6} & \\
\texttt{6 7} & \\
\texttt{6 8} & \\

\hline
\end{tabular}
\end{table}

$1$번 집과 $2$번 집을 잇는 오솔길을 없애면 $\sum d(u,v)=65$,

$2$번 집과 $3$번 집을 잇는 오솔길을 없애면 $\sum d(u,v)=62$,

$3$번 집과 $1$번 집을 잇는 오솔길을 없애면 $\sum d(u,v)=70$이므로

$2$번 집과 $3$번 집을 잇는 오솔길을 없애야한다.